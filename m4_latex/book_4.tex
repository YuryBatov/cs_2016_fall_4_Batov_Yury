\documentclass{book} 

%% Language and font encodings 
\usepackage[english]{babel} 
\usepackage[utf8x]{inputenc} 
\usepackage[T1]{fontenc} 
\usepackage{fancyhdr} 
%% Sets page size and margins 
\usepackage[a4paper,top=3cm,bottom=2cm,left=2.5cm,right=2.5cm,marginparwidth=1.95cm]{geometry} 
\usepackage{color} 
%% Useful packages 
\usepackage[fleqn]{amsmath}
\usepackage{graphicx} 
\usepackage[colorinlistoftodos]{todonotes} 
\usepackage[colorlinks=true, allcolors=blue]{hyperref} 
\usepackage{setspace} 
\definecolor{light-gray}{rgb}{0.8,0.8,0.8}
\setcounter{chapter}{1}
\setcounter{section}{4}
\setlength{\headheight}{12pt}
\newcounter{pro1}
\setcounter{pro1}{5}
\newcommand{\pro}{\par\addtocounter{pro1}{1}
\textbf{Problem \arabic{chapter}.\arabic{pro1} }\quad}
\setcounter{equation}{13}

\begin{document}
\pagestyle{fancy} 
\renewcommand{\headrulewidth}{0pt} 
\fancyhead[RO]{\large \textsl {\textbf{9}}} 
\fancyhead[LO]{\large \textsl{1.3 Guessing integrals}} 
\begin{flushleft}
\Large\textrm{or}
\end{flushleft}
\begin{equation}
  [\alpha]=[x]^{-2}
\end{equation}
\Large\textrm {This conclusion is useful, but continuing to use unspecified but general
dimensions requires lots of notation, and the notation risks burying the
reasoning.}\\
\\
\Large\textrm {The simplest alternative is to make $x$ dimensionless. That choice makes $\alpha$
and $f(\alpha)$ dimensionless, so any candidate for $f(\alpha)$ would be dimensionally
valid, making dimensional analysis again useless. The simplest effective
alternative is to give x simple dimensions—for example, length. (This
choice is natural if you imagine the x axis lying on the floor.) Then
 $[\alpha] = L^{−2}$ .}\\
 \\
 \large\textrm {\textbf{1.3.2 Dimensions of the integral}}\\
 \\
 \Large\textrm{The assignments $[x] = L$ and $[\alpha] = L^{−2}$ determine the dimensions of the
Gaussian integral. Here is the integral again:}\\
\begin{equation}
\int_{-\infty}^{\infty}e^{-\alpha x^{2}}dx.
\end{equation}
\Large\textrm{The dimensions of an integral depend on the dimensions of its three
pieces: the integral sign $\int$,the integrand $e^{-\alpha x^{2}}$, and the differential $dx$}\\
\\
\Large\textrm{The integral sign originated as an elongated S for  \textit{Summe}, the German
word for sum. In a valid sum, all terms have identical dimensions: The
fundamental principle of dimensions requires that apples be added only
to apples. For the same reason, the entire sum has the same dimensions
as any term. Thus, the summation sign—and therefore the integration
sign—do not affect dimensions: The integral sign is dimensionless.}\\
\\
\colorbox{light-gray}{
\begin{minipage}{\textwidth}
\large\textrm {\textbf{\pro Integrating velocity}}\\
\large\textrm {Position is the integral of velocity. However, position and velocity have different
dimensions. How is this difference consistent with the conclusion that the
integration sign is dimensionless?}
\end{minipage}
}
\\
\\
\Large\textrm{Because the integration sign is dimensionless, the dimensions of the integral
are the dimensions of the exponential factor $e^{−\alpha x^{2}}$
multiplied by the
dimensions of $dx$. The exponential, despite its fierce exponent $−\alpha x^{2}$, is
merely several copies of e multiplied together. Because e is dimensionless,
so is $e^{−\alpha x^{2}}$
.}


\newpage
\pagestyle{fancy} 
\renewcommand{\headrulewidth}{0pt} 
\fancyhf{}
\fancyhead[LE]{\large \textsl {\textbf{10}}} 
\fancyhead[RE]{\large \textsl{1 Dimensions}}
\begin{flushleft}
\Large\textit{What are the dimensions of $dx$?}
\end{flushleft}
\Large \textrm{To find the dimensions of dx, follow the advice of Silvanus Thompson
[45, p. 1]: Read d as “a little bit of.” Then dx is “a little bit of x.” A little
length is still a length, so dx is a length. In general, dx has the same
dimensions as x. Equivalently, d—the inverse of $\int$—is dimensionless.}\\
\\
\Large\textrm{Assembling the pieces, the whole integral has dimensions of length:}
\begin{equation}
\biggl[ \int e^{-\alpha x^{2}}dx\biggl]=\underbrace{\Bigl[ e^{-\alpha x^{2}}dx \Bigl]}_l \times \underbrace{[dx]}_L=L
\end{equation}
\colorbox{light-gray}{
\begin{minipage}{\textwidth}
\large\textrm{\textbf{\pro Don’t integrals compute areas?}}\\
\large\textrm{A common belief is that integration computes areas. Areas have dimensions of $L^{2}$. How then can the Gaussian integral have dimensions of L?}
\end{minipage}
}
\\
\\
\\
\Large\textrm{\textbf{1.3.3 Making an $f(\alpha)$ with correct dimensions}}\\
\\
\Large\textrm{The third and final step in this dimensional analysis is to construct an $f(\alpha)
$with the same dimensions as the integral. Because the dimensions of $\alpha$
are $L^{−2}$, the only way to turn $\alpha$ into a length is to form $\alpha^{−1/2}$. Therefore,}
\begin{equation}
f(\alpha)\sim\alpha^{-1/2}.
\end{equation}
\Large\textrm{This useful result, which lacks only a dimensionless factor, was obtained
without any integration.}\\
\Large\textrm{To determine the dimensionless constant, set $\alpha = 1$ and evaluate}
\begin{equation}
f(1)=\int_{-\infty}^{\infty}e^{- x^{2}}dx.
\end{equation}
\Large\textrm{This classic integral will be approximated in Section 2.1 and guessed to be $\sqrt[]{\pi}$.The two results $f(1)=\sqrt[]\pi$ and $f(\alpha)\sim\alpha^{-1/2}$ require that$f(\alpha)=\sqrt[]{\pi/\alpha}$,which yields}
\begin{equation}
\int_{-\infty}^{\infty}e^{-\alpha x^{2}}dx=\sqrt{\frac{\pi}{\alpha}}.
\end{equation}
\Large\textrm{We often memorize the dimensionless constant but forget the power of $\alpha$.
Do not do that. The $\alpha$ factor is usually much more important than the
dimensionless constant. Conveniently, the $\alpha$ factor is what dimensional
analysis can compute.}

\newpage
\pagestyle{fancy} 
\renewcommand{\headrulewidth}{0pt} 
\fancyhf{}
\fancyhead[RO]{\large \textsl {\textbf{11}}} 
\fancyhead[LO]{\large \textsl{1.4 Summary and further problems}} 
\colorbox{light-gray}{
\begin{minipage}{\textwidth}
\large\textrm{\textbf{\pro Change of variable}}\\
\large\textrm{Rewind back to page 8 and pretend that you do not know $f(\alpha)$. Without doing
dimensional analysis, show that $f(\alpha) \sim \alpha^{−1/2}$.}\\
\large\textrm{\textbf{\pro Easy case $\alpha$ = 1}}\\
Setting $\alpha$ = 1, which is an example of easy-cases reasoning (Chapter 2), violates
the assumption that x is a length and $\alpha$ has dimensions of $L^{−2}$. Why is it okay
to set $\alpha$ = 1?\\
\large\textrm{\textbf{\pro Integrating a difficult exponential}}\\
Use dimensional analysis to investigate $\displaystyle\int_{0}^{\infty}e^{-\alpha t^{3}}dt$
\end{minipage}
}
\\
\\
\\
\Large\textrm{\textbf{1.4 Summary and further problems}}\\

\Large\textrm{Do not add apples to oranges: Every term in an equation or sum must
have identical dimensions! This restriction is a powerful tool. It helps us
to evaluate integrals without integrating and to predict the solutions of
differential equations. Here are further problems to practice this tool.}\\
\\
\colorbox{light-gray}{
\begin{minipage}{\textwidth}
\large\textrm{\textbf{\pro Integrals using dimensions}}\\
\large\textrm{Use dimensional analysis to find $\displaystyle\int_{0}^{\infty}e^{-\alpha x}dx$ and $\displaystyle\int \frac{dx}{x^2+a^2}.$A useful result is}
\begin{equation}
\int\frac{dx}{x^2+1}=\arctan x+C.
\end{equation}
\large\textrm{\textbf{\pro Stefan–Boltzmann law}}\\
\large\textrm{Blackbody radiation is an electromagnetic phenomenon, so the radiation intensity
depends on the speed of light $c$.It is also a thermal phenomenon, so it
depends on the thermal energy $k_BT$, where $T$ is the object’s temperature and $k_BT$ is Boltzmann’s constant. And it is a quantum phenomenon, so it depends on
Planck’s constant $\hbar$ . Thus the blackbody-radiation intensity I depends on $c$,$k_BT$,and $\hbar$.Use dimensional analysis to show that I $\alpha$ $ T^4$ and to find the constant
of proportionality $\sigma$.Then look up the missing dimensionless constant. (These
results are used in Section 5.3.3.)}
\\
\large\textrm{\textbf{\pro Arcsine integral}}\\
Use dimensional analysis to find $\displaystyle\int \sqrt[]{1-3x^2}dx.$ A useful result is
\begin{equation}
\int\sqrt[]{1-x^2}dx=\frac{\arcsin x}{2}+\frac{x \sqrt[]{1-x^2}}{2}+C,
\end{equation}
\end{minipage}
}

\newpage
\pagestyle{fancy} 
\renewcommand{\headrulewidth}{0pt} 
\fancyhf{}
\fancyhead[LE]{\large \textsl {\textbf{12}}} 
\fancyhead[RE]{\large \textsl{1 Dimensions}}
\colorbox{light-gray}{
\begin{minipage}{\textwidth}
\begin{flushleft}
\large\textrm{\textbf{\pro Related rates}}\\
Water is poured into a large inverted cone (with a $90^\circ
$ opening
angle) at a rate $\displaystyle dV/dT=10m^3s^-1.$ When the water
depth is h = 5 m, estimate the rate at which the depth is
increasing. Then use calculus to find the exact rate.
\\
\large\textrm{\textbf{\pro Kepler’s third law}}\\
Newton’s law of universal gravitation—the famous inverse-square law—says that
the gravitational force between two masses is
\begin{equation}
F=-\frac{Gm_1m_2}{r^2},
\end{equation}
where $G$ is Newton’s constant, $m_1$ and $m_2$ are the two masses, and $r$ is their
separation. For a planet orbiting the sun, universal gravitation together with
Newton’s second law gives
\begin{equation}
m\frac{d^2\textbf{r}}{dt^2}=-\frac{GMm}{r^2}\hat r,
\end{equation}
where $M$ is the mass of the sun, $m$ the mass of the planet, \textbf{r} is the vector from the sun to the planet, and ${\hat r}$ is the unit vector in the \textbf{r} direction.\\
How does the orbital period $\tau
$ depend on orbital radius r? Look up Kepler’s third law and compare your result to it.
\end{flushleft}
\end{minipage}
}
\end{document}
